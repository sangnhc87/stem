%%%%%%%%%%%%%%%%%%%%%%%%%%%%%%%%%%%%%%%%%%%%%%%%%%%%%%%%%%%%%%%%%
% --- BẢN NÂNG CẤP ĐẲNG CẤP - GIÁO ÁN HỘI THI STEM ---
% --- Tác giả: Gemini (dựa trên file của người dùng) ---
% --- Tương thích PDFLATEX, thiết kế hiện đại, sửa mọi lỗi ---
% --- NỘI DUNG ĐÃ ĐƯỢC CẬP NHẬT ĐỂ KHỚP VỚI WEB APP NÂNG CAO ---
%%%%%%%%%%%%%%%%%%%%%%%%%%%%%%%%%%%%%%%%%%%%%%%%%%%%%%%%%%%%%%%%%
\documentclass[12pt,a4paper]{article}

% --- 1. GÓI CƠ BẢN VÀ NGÔN NGỮ (CHO PDFLATEX) ---
\usepackage[utf8]{inputenc}    % Mã hóa UTF-8 (thay cho vietnam)
\usepackage[T1]{fontenc}       % Mã hóa font chữ
\usepackage[vietnamese]{babel} % Gói ngôn ngữ chuẩn (thay cho vietnam)
\usepackage{graphicx}
\usepackage{amsmath,amssymb}
\usepackage{array}
\usepackage{booktabs}            % Bảng đẹp (không gạch dọc)
\usepackage{multirow}
\usepackage{tabularx}
\usepackage{wrapfig}
\usepackage{lipsum}              % Tạo văn bản mẫu
\usepackage{ragged2e}            % Căn lề
\usepackage{setspace}            % Giãn dòng
\usepackage{tikz}
\usetikzlibrary{shapes,arrows,positioning}
\usepackage{fontawesome5}        % Icon (nếu cần)
\usepackage{verbatim}            % Thêm gói verbatim
\usepackage{colortbl}            % Thêm gói tô màu bảng

% --- 2. CẤU HÌNH FONT (CHO PDFLATEX) ---
% Xóa \fontspec, thay bằng các gói tương đương
\usepackage{newtxtext, newtxmath} % Font Times New Roman
\usepackage[scaled=0.92]{helvet}  % Font Arial/Helvetica
\usepackage{courier}              % Font Courier New
\usepackage{microtype}            % Cải thiện hiển thị

% --- 3. MÀU SẮC VÀ BỐ CỤC ---
\usepackage[svgnames]{xcolor}
\definecolor{primary}{RGB}{0,82,155}     % Xanh đậm
\definecolor{secondary}{RGB}{102,102,102} % Xám
\definecolor{accent}{RGB}{204,0,0}      % Đỏ
\definecolor{success}{RGB}{0,128,0}     % Xanh lá
\definecolor{warning}{RGB}{255,165,0}   % Cam
% SỬA LỖI: Định nghĩa màu TableHeader
\definecolor{TableHeader}{gray}{0.9} 

\usepackage[left=2cm,right=2cm,top=2.5cm,bottom=2cm,headheight=15.6pt]{geometry} % Sửa Headheight
\usepackage{fancyhdr}
\pagestyle{fancy}
\fancyhf{}
\renewcommand{\headrule}{\color{primary}\hrule width\textwidth height 1.5pt}
\fancyhead[L]{\color{primary}\footnotesize\sffamily\textbf{SỞ GD\&ĐT TP.HCM - HỘI THI STEM/STEAM CẤP THÀNH PHỐ}}
\fancyhead[R]{\color{primary}\footnotesize\sffamily\textbf{NĂM HỌC 2025-2026}}
\fancyfoot[C]{\sffamily\thepage}
\renewcommand{\footrule}{\color{primary}\hrule width\textwidth height 0.8pt}

% --- 4. GÓI HỖ TRỢ VÀ TÙY CHỈNH ---
\usepackage{enumitem}
\usepackage{tcolorbox}
\tcbuselibrary{skins, breakable}
\usepackage[
    colorlinks=true,
    linkcolor=primary,
    urlcolor=accent,
    citecolor=success,
    pdftitle={Giao an du thi STEM: Tu van Tiet kiem},
    pdfauthor={Nguyen Van A} % Tên từ file
]{hyperref}

% --- 5. ĐỊNH DẠNG TIÊU ĐỀ (GIỮ NGUYÊN) ---
\usepackage{titlesec}
% Bỏ \titleformat{\section} vì đã dùng \competitionsection
\titleformat{\subsection}
{\normalfont\bfseries\color{accent}\sffamily}
{\llap{\color{accent}\rule{1.2em}{1.2em}\hspace{0.5em}}}
{0em}
{}

\titleformat{\subsubsection}
{\normalfont\bfseries\itshape\color{secondary}\sffamily}
{\llap{\color{secondary}\rule{1em}{1em}\hspace{0.5em}}}
{0em}
{}

% --- 6. MÔI TRƯỜNG TÙY CHỈNH (NÂNG CẤP VÀ SỬA LỖI) ---

% Tiêu đề cho các phần
\def\phanI{TỔNG QUAN}
\def\phanII{CHUẨN BỊ CỦA GIÁO VIÊN}
\def\phanIII{KẾ HOẠCH DẠY HỌC ĐỀ XUẤT}
\def\phanIV{HƯỚNG DẪN HỌC SINH}
\def\phanV{CÁC PHỤ LỤC}

% SỬA LỖI: Định nghĩa môi trường 'phan' bằng \newtcolorbox
\newtcolorbox{competitionsection}[1]{
    enhanced, 
    breakable, % Cho phép ngắt trang
    colback=white, 
    colframe=primary, 
    boxrule=0.8pt,
    fonttitle=\bfseries\sffamily\Huge, 
    coltitle=white, 
    colbacktitle=primary,
    % Đặt tiêu đề như 1 cái tab
    attach boxed title to top left={xshift=1cm, yshift=-1mm-\tcboxedtitleheight/2},
    boxed title style={sharp corners, boxrule=0pt},
    title={PHẦN #1: \csname phan#1\endcsname},
    % Tự động thêm vào Mục lục
    before upper={
        \addcontentsline{toc}{section}{Phần #1: \csname phan#1\endcsname}
        %\clearpage % Tự động sang trang mới cho mỗi phần
    }
}


% Định dạng bảng hiện đại
\newcolumntype{C}{>{\centering\arraybackslash}X}
\newcolumntype{L}{>{\raggedright\arraybackslash}X}
\newcolumntype{R}{>{\raggedleft\arraybackslash}X}
% SỬA LỖI: Định nghĩa \header
\newcommand{\header}[1]{\cellcolor{TableHeader}\sffamily\bfseries#1}

% Bảng thời gian giảng dạy
\newcommand{\teachingplan}[4]{
\noindent
\begin{minipage}{0.15\textwidth}
\centering\bfseries\sffamily\color{primary}#1
\end{minipage}%
\begin{minipage}{0.25\textwidth}
\sffamily\textbf{#2}
\end{minipage}%
\begin{minipage}{0.3\textwidth}
\textit{#3}
\end{minipage}%
\begin{minipage}{0.25\textwidth}
\small #4
\end{minipage}
\vspace{1ex}\par\noindent\color{primary!30}\rule{\textwidth}{0.4pt}\vspace{1ex}
}

% Môi trường case study
\newtcolorbox{casestudy}{
    enhanced, breakable,
    colback=primary!5, colframe=primary!80,
    boxrule=1pt, arc=2mm,
    title=\sffamily\bfseries\color{accent}TÌNH HUỐNG THỰC TIỄN,
    fonttitle=\bfseries\sffamily
}

% Môi trường công thức quan trọng
\newtcolorbox{keyformula}{
    enhanced, breakable,
    colback=accent!5, colframe=accent!80,
    boxrule=1pt, arc=2mm,
    title=\sffamily\bfseries\color{primary}CÔNG THỨC TRỌNG TÂM,
    fonttitle=\bfseries\sffamily
}

% Môi trường nhiệm vụ
\newtcolorbox{task}[1]{
    enhanced, breakable,
    colback=warning!10, colframe=warning!80,
    boxrule=1pt, arc=2mm,
    title=\sffamily\bfseries\color{secondary}NHIỆM VỤ: #1,
    fonttitle=\bfseries\sffamily
}

% Môi trường phiếu học tập
\newtcolorbox{phieuhoc_tap_box}[1]{
    enhanced, breakable,
    colback=white, colframe=DarkGreen,
    boxrule=1pt, arc=2mm,
    title=\sffamily\bfseries #1,
    fonttitle=\bfseries\sffamily,
    coltitle=DarkGreen
}


% Định dạng list đẹp hơn (GIỮ NGUYÊN)
\setlist[itemize,1]{label=\textbullet, leftmargin=*, nosep}
\setlist[itemize,2]{label=--, leftmargin=*, nosep}
\setlist[description]{font=\bfseries\sffamily\color{primary}, nosep} % Sửa lỗi dùng \item

\begin{document}

% Trang bìa chuyên nghiệp
\begin{titlepage}
\begin{tikzpicture}[remember picture, overlay]
\node[rectangle, fill=primary, anchor=north, minimum width=\paperwidth, minimum height=2cm] (header) at (current page.north){};
\node[anchor=south, yshift=2mm, text=white] at (header.south) {\Huge\sffamily\bfseries GIÁO ÁN DỰ THI STEM/STEAM CẤP THÀNH PHỐ};
\end{tikzpicture}

\vspace{2cm}
\begin{center}
{\LARGE\bfseries\color{primary}\sffamily SỞ GIÁO DỤC VÀ ĐÀO TẠO THÀNH PHỐ HỒ CHÍ MINH}\\
\vspace{0.5cm}
{\Large\bfseries\sffamily HỘI THI THIẾT KẾ CHỦ ĐỀ DẠY HỌC STEM/STEAM}\\
\vspace{0.3cm}
{\large\sffamily Năm học 2025 - 2026}

\vspace{2cm}

\begin{tcolorbox}[
    width=0.9\textwidth, 
    colback=gray!5, 
    colframe=primary,
    arc=2mm, boxrule=1pt
]
    \centering
    {\LARGE\bfseries\sffamily CHỦ ĐỀ:} \\
    {\Huge\bfseries\color{accent}\sffamily XÂY DỰNG CÔNG CỤ TƯ VẤN TIẾT KIỆM TƯƠNG LAI}\\
    \vspace{0.25cm}
    {\Large\sffamily ỨNG DỤNG HÀM SỐ MŨ VÀ CÔNG NGHỆ SỐ}
\end{tcolorbox}

\vspace{1.5cm}
{\Large\bfseries\sffamily MÔN: TOÁN - KHỐI: 11}

\vfill
\large
% SỬA LỖI: Điền thông tin vào \tabular
\begin{tabular}{rl}
\sffamily\bfseries Giáo viên: & Nguyễn Văn Sang\\
\sffamily\bfseries Đơn vị: & Trường THPT Nguyễn Hữu Cảnh\\
\sffamily\bfseries Số điện thoại: & 0389\,821\,115\\
\end{tabular}

\vspace{1cm}
{\large\sffamily Thành phố Hồ Chí Minh, tháng 11 năm 2025}
\end{center}
\end{titlepage}

% Mục lục
\tableofcontents
\thispagestyle{empty}
% \newpage % Bỏ vì \begin{competitionsection} đã có \clearpage

%%%%% PHẦN 1: TỔNG QUAN %%%%%
\begin{competitionsection}{I}
\vspace{0.5cm}
\subsection*{1.1. Thông tin cá nhân}
% SỬA LỖI: Dùng itemize
\begin{itemize}[leftmargin=*,noitemsep]
\item \textbf{Họ và tên:} Nguyễn Văn Sang
\item \textbf{Chuyên môn:} Toán học
\item \textbf{Thâm niên:} 15 năm giảng dạy THPT
\item \textbf{Đơn vị:} Trường THPT Nguyễn Hữu Cảnh, TP.HCM

\end{itemize}

\subsection*{1.2. Giới thiệu chủ đề}
\begin{itemize}[leftmargin=*,noitemsep]
\item \textbf{Tên chủ đề:} \textcolor{accent}{Xây dựng công cụ tư vấn tiết kiệm tương lai: Ứng dụng hàm số mũ và công nghệ số}
\item \textbf{Bối cảnh thực tiễn:}
    \begin{itemize}
    \item \textbf{76\%} học sinh THPT chưa có kế hoạch tài chính cá nhân (Khảo sát Sở GD\&ĐT TP.HCM 2024)
    \item Trung bình mỗi học sinh mong muốn mua xe máy điện trị giá 15-25 triệu đồng khi vào đại học
    \item Nhu cầu ứng dụng toán học vào giải quyết vấn đề thực tiễn ngày càng cấp thiết trong chương trình GDPT 2018
    \end{itemize}
\item \textbf{Điểm mới và sáng tạo:}
    \begin{itemize}
    \item \textbf{Đa ngành:} Kết hợp Toán học - Kinh tế - Công nghệ thông tin tạo thành vòng tròn STEAM hoàn chỉnh
    % CẬP NHẬT: Khớp với file HTML
    \item \textbf{Sản phẩm thực:} Công cụ web (file HTML) hoàn thiện có thể sử dụng ngay trong đời sống, không cần server.
    \item \textbf{Giáo dục tài chính:} Phát triển năng lực tài chính cá nhân - kỹ năng sống còn thiếu trong chương trình hiện tại
    \item \textbf{Cá nhân hóa:} Mỗi học sinh tạo ra công cụ phù hợp với điều kiện và mục tiêu riêng
    \end{itemize}
\end{itemize}

\subsection*{1.3. Căn cứ thiết kế}
\begin{minipage}{0.48\textwidth}
\textbf{Căn cứ pháp lý:}
\begin{itemize}[leftmargin=*,noitemsep]
\item Chương trình GDPT 2018 môn Toán
\item Công văn 5555/BGDĐT-GDTrH về dạy học STEM
\item Kế hoạch số 4500/KH-SGDĐT của Sở GD\&ĐT TP.HCM (Giả định)
\end{itemize}
\end{minipage}
\hfill
\begin{minipage}{0.48\textwidth}
\textbf{Cơ sở lý luận:}
\begin{itemize}[leftmargin=*,noitemsep]
\item Lý thuyết học tập qua dự án (Project-Based Learning)
\item Mô hình 5E (Engage, Explore, Explain, Elaborate, Evaluate)
\item Khung năng lực tài chính của OECD
\end{itemize}
\end{minipage}

\subsection*{1.4. Mục tiêu giáo dục}
% SỬA LỖI: Bảng không thể chứa list, dùng \parbox
\begin{center}
\begin{tabularx}{\linewidth}{@{} L L L @{}}
\toprule
\header{Kiến thức} & \header{Kỹ năng} & \header{Thái độ} \\
\midrule
% Dùng \parbox để chứa list
\parbox[t]{\dimexpr1.\linewidth-2\tabcolsep}{
    \begin{itemize}[leftmargin=*,noitemsep,topsep=1ex, partopsep=0pt]
        \item Nắm vững công thức lãi kép
        \item Hiểu bản chất của hàm số mũ trong tài chính
        % CẬP NHẬT: Thêm Lạm phát (có trong app)
        \item Hiểu khái niệm lạm phát
    \end{itemize}
} &
\parbox[t]{\dimexpr1.\linewidth-2\tabcolsep}{
    \begin{itemize}[leftmargin=*,noitemsep,topsep=1ex, partopsep=0pt]
        \item Mô hình hóa toán học
        \item Sử dụng công nghệ số để tự động hóa
        % CẬP NHẬT: Kỹ năng có trong app
        \item Phân tích kịch bản và độ nhạy
        \item Đọc hiểu biểu đồ (tròn, cột, đường)
    \end{itemize}
} &
\parbox[t]{\dimexpr1.\linewidth-2\tabcolsep}{
    \begin{itemize}[leftmargin=*,noitemsep,topsep=1ex, partopsep=0pt]
        \item Ý thức tiết kiệm
        \item Tư duy phản biện
        \item Trách nhiệm với tài chính cá nhân
        \item Tôn trọng sự khác biệt
    \end{itemize}
} \\
\bottomrule
\end{tabularx}
\end{center}

\subsection*{1.5. Cấu trúc chương trình}
\begin{center}
\begin{tikzpicture}[node distance=1.5cm, every node/.style={font=\sffamily}]
\node (start) [rectangle, rounded corners, fill=primary!20, text width=4cm, align=center, minimum height=1.2cm] {\textbf{Khám phá thực tiễn}\\ Tình huống: Mua xe máy điện};
\node (mid1) [rectangle, rounded corners, fill=accent!20, text width=4cm, align=center, below of=start, yshift=-1.2cm, minimum height=1.2cm] {\textbf{Xây dựng mô hình toán học}\\ Công thức lãi kép};
\node (mid2) [rectangle, rounded corners, fill=primary!20, text width=4cm, align=center, below of=mid1, yshift=-1.2cm, minimum height=1.2cm] {\textbf{Phát triển công cụ số}\\ Ứng dụng web (HTML/JS)};
\node (end) [rectangle, rounded corners, fill=accent!20, text width=4cm, align=center, below of=mid2, yshift=-1.2cm, minimum height=1.2cm] {\textbf{Ứng dụng thực tế}\\ Lập kế hoạch tài chính cá nhân};
\draw[->, thick, color=primary] (start) -- (mid1) node[midway, right, text width=3cm, align=center] {\footnotesize \textbf{M} (Toán học)};
\draw[->, thick, color=primary] (mid1) -- (mid2) node[midway, right, text width=3cm, align=center] {\footnotesize \textbf{T} (Công nghệ)};
\draw[->, thick, color=primary] (mid2) -- (end) node[midway, right, text width=3cm, align=center] {\footnotesize \textbf{A} (Kinh tế)};
\end{tikzpicture}
\end{center}
\end{competitionsection}
\newpage
%%%%% PHẦN 2: CHUẨN BỊ %%%%%
\begin{competitionsection}{II}
\vspace{0.5cm}
\subsection*{2.1. Thiết bị và công nghệ}
\begin{itemize}[leftmargin=*,noitemsep]
\item \textbf{Phần cứng:}
    \begin{itemize}
    \item Máy tính cấu hình tối thiểu 4GB RAM, kết nối Internet (1 máy/2 HS)
    \item Máy chiếu tương tác thông minh + màn hình cảm ứng
    \item Máy tính cá nhân của học sinh (BYOD - Bring Your Own Device)
    \end{itemize}
\item \textbf{Phần mềm và nền tảng:}
    % CẬP NHẬT: Khớp với file HTML
    \begin{itemize}
    \item \textbf{Học liệu số chính:} Ứng dụng web \texttt{CongCuTuVanTietKiem.html} (đính kèm) hoặc truy cập website \href{https://stem-1h8.pages.dev/}{stem-1h8.pages.dev}.
    \item \textit{Yêu cầu:} Trình duyệt web (Chrome, Firefox, Edge...)
    \item Google Sheets template (dự phòng)
    \item Padlet để chia sẻ kết quả nhóm
    \item Kahoot! để kiểm tra nhanh kiến thức
    \end{itemize}
\item \textbf{Tài liệu:}
    \begin{itemize}
    \item Phiếu học tập số 1: Khám phá công thức lãi kép
    % CẬP NHẬT: PHT 2 phải khớp với app
    \item Phiếu học tập số 2: Hướng dẫn sử dụng 3 tab của Web App
    \item Bảng tham khảo lãi suất cập nhật tháng 10/2025
    \item Rubric đánh giá sản phẩm
    \end{itemize}
\end{itemize}

\subsection*{2.2. Kịch bản giảng dạy}
\begin{center}
\begin{tabularx}{\linewidth}{@{} L L L L L @{}}
\toprule
\header{Thời lượng} & \header{Hoạt động} & \header{Phương pháp} & \header{Công cụ} & \header{Đánh giá} \\
\midrule
10' & Khởi động: Video case study & Tình huống thực tiễn & Clip phỏng vấn HS & Quan sát mức độ hứng thú \\
15' & Khám phá công thức & Khám phá có hướng dẫn & Phiếu học tập số 1 & Phiếu trả lời nhanh \\
% CẬP NHẬT: "Demo ứng dụng (3 tabs)"
\rowcolor{primary!10}
10' & Chuyển tiếp sang công nghệ & Đàm thoại & Demo ứng dụng (3 tabs) & Câu hỏi phản hồi \\
\rowcolor{primary!10}
20' & Thực hành (Tab 1, 2) & Học qua dự án & Ứng dụng web & Đánh giá quá trình \\
\rowcolor{primary!10}
15' & Phân tích kết quả & Hợp tác nhóm & Bảng so sánh (Tab 2) & Peer assessment \\
10' & Trình bày sản phẩm & Thuyết trình & Google Slides & Rubric đánh giá \\
5' & Tổng kết mở rộng & Suy ngẫm & Mindmap số (Tab 3) & Nhật ký học tập \\
5' & Giao bài tập về nhà & Cá nhân hóa & LMS của trường & Đánh giá sản phẩm \\
\bottomrule
\end{tabularx}
\end{center}

\subsection*{2.3. Đánh giá đa chiều}
\begin{minipage}[t]{0.48\textwidth}
\textbf{Đánh giá quá trình:}
\begin{itemize}[leftmargin=*,noitemsep]
\item \textbf{Phiếu quan sát nhóm:} Ghi nhận mức độ hợp tác, giải quyết vấn đề
\item \textbf{Nhật ký học tập điện tử:} Ghi chép suy nghĩ, thắc mắc của học sinh
\item \textbf{Phiếu phản hồi nhanh:} Google Form thu thập phản hồi tức thì
\item \textbf{Đánh giá đồng cấp:} Học sinh đánh giá lẫn nhau trong nhóm
\end{itemize}
\end{minipage}
\hfill
\begin{minipage}[t]{0.48\textwidth}
\textbf{Đánh giá sản phẩm:}
\begin{itemize}[leftmargin=*,noitemsep]
\item \textbf{Rubric 4 tiêu chí:} (Xem Phụ lục 3)
    \begin{itemize}
    \item Tính chính xác của công thức (4 điểm)
    % CẬP NHẬT: Tiêu chí khớp với app
    \item Khả năng sử dụng công nghệ (3 tabs) (4 điểm)
    \item Tính khả thi của phương án (4 điểm)
    \item Khả năng trình bày và bảo vệ ý tưởng (4 điểm)
    \end{itemize}
\item \textbf{Sản phẩm demo ứng dụng:} Đánh giá tính thực tiễn và sáng tạo
\item \textbf{Bài tập về nhà:} Áp dụng công cụ cho mục tiêu cá nhân
\end{itemize}
\end{minipage}

\subsection*{2.4. Kế hoạch dự phòng}
\begin{itemize}[leftmargin=*,noitemsep]
\item \textbf{Tình huống 1:} Mất điện hoặc mất kết nối Internet
    \begin{itemize}
    \item Học liệu số \texttt{CongCuTuVanTietKiem.html} \textbf{không cần Internet}, chỉ cần mở bằng trình duyệt.
    \item Nếu mất điện: Chuyển sang sử dụng Google Sheets offline (nếu máy tính còn pin) hoặc phiếu tính toán bằng tay.
    \end{itemize}
\item \textbf{Tình huống 2:} Học sinh không có thiết bị cá nhân
    \begin{itemize}
    \item Sắp xếp nhóm 3-4 học sinh/1 máy tính của trường
    \item Sử dụng phương pháp "station rotation" để luân phiên sử dụng thiết bị
    \end{itemize}
\item \textbf{Tình huống 3:} Một số học sinh tiếp thu nhanh hơn
    \begin{itemize}
    % CẬP NHẬT: Khớp với app
    \item Giao nhiệm vụ mở rộng: Khám phá "Tab Học tập" và "Phân tích độ nhạy".
    \item Giao vai trò hỗ trợ các nhóm khác
    \end{itemize}
\end{itemize}
\end{competitionsection}

%%%%% PHẦN 3: KẾ HOẠCH DẠY HỌC %%%%%
\begin{competitionsection}{III}
    \vspace{0.5cm}
\subsection*{3.1. Tiến trình hai tiết học (Tóm tắt)}
\teachingplan{10'}{Khởi động}{Video tình huống: Nam (16t) muốn mua xe máy điện 20 triệu}{Công nghệ: Clip tương tác}
\teachingplan{15'}{Khám phá toán học}{Xây dựng công thức lãi kép}{PP: Khám phá có hướng dẫn}
% CẬP NHẬT: Khớp với app
\teachingplan{10'}{Chuyển tiếp}{Demo ứng dụng web (3 tabs)}{Công nghệ: Màn hình tương tác}
\teachingplan{20'}{Thực hành}{Sử dụng Tab 1 (Tính toán) và Tab 2 (Kịch bản)}{PP: Học qua dự án}
\teachingplan{15'}{Phân tích}{So sánh kịch bản, phân tích độ nhạy}{Công cụ: Biểu đồ cột, đường}
\teachingplan{10'}{Báo cáo}{Thuyết trình nhóm}{Công nghệ: Google Slides chia sẻ}
\teachingplan{5'}{Tổng kết}{Hệ thống hóa kiến thức (Tab 3)}{PP: Sơ đồ tư duy}
\teachingplan{5'}{Mở rộng}{Bài tập về nhà ứng dụng}{Nền tảng: Google Classroom}

\subsection*{3.2. Kịch bản chi tiết}

\begin{casestudy}
Nam là học sinh lớp 11, mong muốn mua một chiếc xe máy điện trị giá 20.000.000 VNĐ khi vào đại học (sau 2 năm nữa). Hiện tại Nam được gia đình cho 5.000.000 VNĐ. Mỗi tháng Nam có thể tiết kiệm từ 300.000 - 500.000 VNĐ từ tiền tiêu vặt và làm thêm. Với các mức lãi suất tiết kiệm khác nhau, Nam nên chọn phương án nào để đạt được mục tiêu?
\end{casestudy}

\subsubsection*{Hoạt động 1: Khởi động - Kết nối thực tiễn (10 phút)}
\begin{itemize}[leftmargin=*,noitemsep]
\item \textbf{Bước 1:} Giáo viên chiếu video phỏng vấn 3 học sinh về ước mơ mua xe máy điện khi vào đại học
\item \textbf{Bước 2:} Đặt câu hỏi kích thích tư duy:
    \begin{itemize}
    \item \textit{"Theo em, việc để dành tiền mua xe sau 2 năm có dễ dàng không?"}
    \item \textit{"Làm thế nào để số tiền tiết kiệm sinh lời hiệu quả?"}
    \end{itemize}
\item \textbf{Bước 3:} Học sinh thảo luận cặp đôi (2 phút) và chia sẻ ý tưởng ban đầu
\item \textbf{Dự kiến câu trả lời:}
    \begin{itemize}
    \item Gửi ngân hàng để hưởng lãi
    \item Gửi càng sớm càng tốt để tiền sinh lời nhiều hơn
    \item Cần tính toán chi tiết để tránh thiếu hụt
    \end{itemize}
\item \textbf{Chuyển tiếp:} \textit{"Hôm nay chúng ta sẽ sử dụng toán học và công nghệ để xây dựng công cụ giúp Nam và các bạn có quyết định tài chính sáng suốt."}
\end{itemize}

\subsubsection*{Hoạt động 2: Khám phá công thức toán học (15 phút)}
\begin{task}{Nhóm}
\textbf{Nhiệm vụ:} Khám phá công thức lãi kép thông qua tình huống cụ thể (PHT 1)
\textbf{Hướng dẫn:}
\begin{itemize}[leftmargin=*,noitemsep]
\item Tính số tiền nhận được sau 1, 2, 3 tháng nếu gửi 5.000.000 VNĐ với lãi suất 0,583\%/tháng (tương đương 7\%/năm)
\item Nhận xét quy luật và tổng quát thành công thức
\item Thử áp dụng công thức cho trường hợp gửi tiền hàng tháng
\end{itemize}
\end{task}

\begin{keyformula}
\textbf{Công thức 1: Lãi kép (gửi một lần)}
$$A = P \times (1 + r)^n$$
\textbf{Công thức 2: Lãi kép (gửi định kỳ)}
$$A = P \times \frac{(1+r)^n-1}{r}$$
Trong đó:
\begin{itemize}[leftmargin=*,noitemsep]
\item $A$: Số tiền nhận được sau $n$ kỳ
\item $P$: Tiền gốc (hoặc số tiền gửi mỗi kỳ)
\item $r$: Lãi suất theo kỳ (tháng)
\item $n$: Số kỳ tính lãi
\end{itemize}
\end{keyformula}

% CẬP NHẬT: Hoạt động 3 khớp với 3 tab của app
\subsubsection*{Hoạt động 3: Chuyển tiếp sang công nghệ (10 phút)}
\begin{itemize}[leftmargin=*,noitemsep]
\item \textbf{Bước 1:} Giáo viên demo ứng dụng web \texttt{CongCuTuVanTietKiem.html} (PHT 2).
\item \textbf{Bước 2:} Thao tác minh họa 3 tab chính:
    \begin{itemize}
    \item \textbf{Tab 1 - Tính toán:} Nhập các tham số (5tr, 400k, 7\%, 24 tháng). Cho thấy kết quả (16.05tr) và biểu đồ tròn. Nhấn bật "Lạm phát" để thấy giá trị thực bị giảm.
    \item \textbf{Tab 2 - Phân tích kịch bản:} Thêm 2-3 kịch bản (VD: tăng tiền, tăng thời gian). Cho thấy biểu đồ cột so sánh. Kéo xuống cho thấy biểu đồ đường "Phân tích độ nhạy".
    \item \textbf{Tab 3 - Học tập:} Lướt qua nội dung công thức, các kênh đầu tư và quy trình STEM.
    \end{itemize}
\item \textbf{Bước 3:} Đặt câu hỏi chuyển tiếp:
    \begin{itemize}
    \item \textit{"Làm thế nào để nhanh chóng so sánh nhiều phương án khác nhau?"}
    \item \textit{"Yếu tố nào (tiền gửi hay lãi suất) ảnh hưởng đến kết quả nhiều hơn?"}
    \end{itemize}
\end{itemize}

% CẬP NHẬT: Hoạt động 4 khớp với app
\subsubsection*{Hoạt động 4: Xây dựng công cụ tính toán (20 phút)}
\begin{task}{Nhóm}
\textbf{Nhiệm vụ:} Sử dụng Web App (PHT 2) để tìm 2 phương án khả thi đạt mục tiêu 20 triệu.
\textbf{Yêu cầu cụ thể:}
\begin{itemize}[leftmargin=*,noitemsep]
\item \textbf{Tab 1:} Nhập kịch bản gốc, ghi lại kết quả (chưa đạt).
\item \textbf{Tab 2:} Sử dụng nút "Thêm kịch bản" để thử nghiệm các thay đổi (tăng tiền gửi, tăng thời gian, tìm lãi suất cao hơn).
\item \textbf{Tab 2:} Tìm ra 2 kịch bản đạt được mục tiêu 20 triệu. Chụp màn hình (hoặc dùng nút "Xuất biểu đồ").
\item \textbf{Tab 2:} Quan sát biểu đồ "Phân tích độ nhạy" và nhận xét.
\end{itemize}
\end{task}
\begin{itemize}[leftmargin=*,noitemsep]
\item \textbf{Hỗ trợ giáo viên:}
    \begin{itemize}
    \item Tuần tra các nhóm, hỗ trợ kỹ thuật khi cần
    \item Gợi ý các ngân hàng có lãi suất cạnh tranh (xem Phụ lục 4)
    \item Hướng dẫn cách đọc biểu đồ cột (so sánh) và biểu đồ đường (độ nhạy).
    \end{itemize}
\end{itemize}

% CẬP NHẬT: Hoạt động 5 khớp với app
\subsubsection*{Hoạt động 5: Phân tích và trình bày (15 phút)}
\begin{itemize}[leftmargin=*,noitemsep]
\item \textbf{Bước 1:} 2 nhóm được chọn ngẫu nhiên trình bày phương án của mình (3 phút/nhóm) (dựa trên PHT 2).
\item \textbf{Bước 2:} Các nhóm khác đặt câu hỏi phản biện
\item \textbf{Bước 3:} Giáo viên đặt câu hỏi mở rộng:
    \begin{itemize}
    % CẬP NHẬT: Câu hỏi Lạm phát khớp với tính năng của app
    \item \textit{"Sử dụng \textbf{Tab Tính toán}, hãy bật chức năng 'Lạm phát' (với tỉ lệ 3.5\%). Giá trị thực của tiền tiết kiệm thay đổi như thế nào so với giá trị danh nghĩa?"}
    \item \textit{"Dựa trên biểu đồ 'Phân tích độ nhạy' ở \textbf{Tab 2}, yếu tố nào (tăng tiền gửi hay tăng lãi suất) có ảnh hưởng mạnh mẽ hơn?"}
    \end{itemize}
\item \textbf{Bước 4:} Cả lớp thảo luận và thống nhất phương án tối ưu
\end{itemize}

\subsubsection*{Hoạt động 6: Tổng kết và mở rộng (10 phút)}
\begin{itemize}[leftmargin=*,noitemsep]
\item \textbf{Tổng kết kiến thức:} Giáo viên hệ thống hóa (sử dụng \textbf{Tab 3 - Học tập}):
    \begin{itemize}
    \item Công thức lãi kép là ứng dụng thực tế của hàm số mũ (M).
    \item Công nghệ (T) giúp tự động hóa, trực quan hóa (biểu đồ) và phân tích (độ nhạy).
    \item Lập kế hoạch tài chính (A - Arts/Kinh tế) từ sớm mang lại lợi ích lâu dài.
    \end{itemize}
\item \textbf{Bài tập về nhà:}
    \begin{itemize}
    \item Sử dụng công cụ (Web App) đã học để lập kế hoạch tài chính cho một mục tiêu cá nhân.
    \item Trình bày trong 1 trang giấy hoặc slide, bao gồm:
        \begin{itemize}
        \item Mục tiêu cụ thể (số tiền, thời gian)
        \item Phương án tiết kiệm được lựa chọn (ảnh chụp từ Tab 2)
        \item Lý do lựa chọn và tính khả thi
        \end{itemize}
    \end{itemize}
\item \textbf{Liên kết thực tế:}
    \begin{itemize}
    \item Giới thiệu các kênh đầu tư khác (có trong Tab 3).
    \item Khuyến khích mở tài khoản tiết kiệm học đường tại ngân hàng.
    \end{itemize}
\end{itemize}
\end{competitionsection}

%%%%% PHẦN 4: HƯỚNG DẪN HỌC SINH %%%%%
\begin{competitionsection}{IV}
\vspace{0.5cm}
\subsection*{4.1. Hướng dẫn chi tiết cho học sinh}
\subsubsection*{Giai đoạn 1: Khám phá công thức toán học (PHT 1)}
\begin{itemize}[leftmargin=*,noitemsep]
\item \textbf{Bước 1:} Tính toán số tiền nhận được theo từng tháng:
    \begin{itemize}
    \item Tháng 1: $5.000.000 \times (1 + 0,00583) = 5.029.150$ VNĐ
    \item Tháng 2: $5.029.150 \times (1 + 0,00583) = 5.058.471$ VNĐ
    \item Tháng 3: $5.058.471 \times (1 + 0,00583) = 5.087.965$ VNĐ
    \end{itemize}
\item \textbf{Bước 2:} Nhận xét quy luật:
    \begin{itemize}
    \item Số tiền tăng theo cấp số nhân với công bội $(1 + r)$
    \item Tổng quát công thức: $A = P \times (1 + r)^n$
    \end{itemize}
\item \textbf{Bước 3:} Kiểm tra công thức bằng cách tính cho $n = 24$ tháng
\end{itemize}

% CẬP NHẬT: Giai đoạn 2 khớp với 3 tab của app
\subsubsection*{Giai đoạn 2: Sử dụng công cụ công nghệ (Web App \texttt{CongCuTuVanTietKiem.html})}
\begin{itemize}[leftmargin=*,noitemsep]
\item \textbf{Bước 1: Mở ứng dụng}
    \begin{itemize}
    \item Mở file \texttt{CongCuTuVanTietKiem.html} bằng trình duyệt web.
    \end{itemize}
\item \textbf{Bước 2: Tab 1 - Tính toán}
    \begin{itemize}
    \item Nhập các tham số của bài toán gốc (5tr, 400k, 7\%, 24 tháng).
    \item Ghi lại kết quả (chưa đạt mục tiêu 20tr).
    \item Quan sát biểu đồ tròn (Doughnut chart) để xem cơ cấu tiền.
    \item \textbf{Thử nghiệm:} Bật chức năng "Lạm phát" và xem "giá trị thực" thay đổi thế nào.
    \end{itemize}
\item \textbf{Bước 3: Tab 2 - Phân tích kịch bản}
    \begin{itemize}
    \item Nhấn nút "Thêm kịch bản" để tạo phương án mới.
    \item Thay đổi các tham số (VD: Tăng tiền gửi tháng, tăng thời gian, tăng lãi suất).
    \item Quan sát biểu đồ cột để so sánh các kịch bản.
    \item Kéo xuống dưới, quan sát biểu đồ đường "Phân tích độ nhạy".
    \end{itemize}
\item \textbf{Bước 4: Tab 3 - Học tập}
    \begin{itemize}
    \item Đọc lại 2 công thức lãi kép.
    \item Đọc phần "Phân tích đầu tư" để biết thêm về các kênh khác.
    \item Đọc phần "Quy trình STEM" để hiểu rõ dự án.
    \end{itemize}
\end{itemize}

\subsubsection*{Giai đoạn 3: Phân tích và trình bày}
\begin{itemize}[leftmargin=*,noitemsep]
\item \textbf{Bước 1:} So sánh các phương án (từ Tab 2) dựa trên tiêu chí:
    \begin{itemize}
    \item Khả năng thực hiện (số tiền gửi hàng tháng có phù hợp với điều kiện cá nhân)
    \item Hiệu quả tài chính (tổng số tiền nhận được)
    \item Rủi ro (độ ổn định của lãi suất)
    \end{itemize}
\item \textbf{Bước 2:} Chọn 1 phương án tối ưu và lý giải lựa chọn
\item \textbf{Bước 3:} Chuẩn bị phần trình bày 3 phút với cấu trúc:
    \begin{itemize}
    \item Giới thiệu tình huống và mục tiêu
    \item Trình bày các phương án đã thử nghiệm (chụp ảnh biểu đồ cột từ Tab 2)
    \item Phân tích và lựa chọn phương án tối ưu
    \item Kết luận (có nhắc đến lạm phát và độ nhạy)
    \end{itemize}
\end{itemize}

\subsection*{4.2. Gợi ý mở rộng}
\begin{itemize}[leftmargin=*,noitemsep]
\item \textbf{Tìm hiểu sâu về tài chính cá nhân (từ Tab 3):}
    \begin{itemize}
    \item Nghiên cứu các hình thức đầu tư khác: chứng khoán, bất động sản, vàng
    \item So sánh rủi ro và lợi nhuận của các kênh đầu tư
    \item Tìm hiểu về lạm phát và ảnh hưởng đến giá trị tiền tệ
    \end{itemize}
\item \textbf{Phát triển công cụ (từ Phụ lục 5):}
    \begin{itemize}
    \item Thử "View Source" (Xem nguồn trang) file HTML để đọc code JavaScript.
    \item Đề xuất cải tiến giao diện hoặc thuật toán tính toán
    \item Xây dựng phiên bản dành cho điện thoại di động
    \end{itemize}
\item \textbf{Ứng dụng thực tế:}
    \begin{itemize}
    \item Tạo kế hoạch tài chính cho cả gia đình
    \item Tham gia cuộc thi "Quản lý tài chính học đường" do ngân hàng tổ chức
    \item Viết bài chia sẻ kinh nghiệm trên blog cá nhân hoặc mạng xã hội
    \end{itemize}
\end{itemize}
\end{competitionsection}
\newpage
%%%%% PHẦN 5: CÁC PHỤ LỤC %%%%%
\begin{competitionsection}{V}
\vspace{0.5cm}
\subsection*{Phụ lục 1: Phiếu học tập số 1 - Khám phá công thức lãi kép}
\begin{phieuhoc_tap_box}{Phụ lục 1: Phiếu học tập số 1 - Công thức lãi kép}
\textbf{Bài toán 1:} Gửi 5.000.000 VNĐ vào ngân hàng với lãi suất 0,583\%/tháng (tương đương 7\%/năm). Tính số tiền nhận được sau:
\begin{itemize}[leftmargin=*,noitemsep]
\item 1 tháng: $5.000.000 \times (1 + 0,00583) = \underline{\hspace{3cm}}$ VNĐ
\item 2 tháng: $\underline{\hspace{3cm}} \times (1 + 0,00583) = \underline{\hspace{3cm}}$ VNĐ
\item 3 tháng: $\ldots$
\item n tháng: $\ldots$
\end{itemize}
\textbf{Kết luận:} Công thức lãi kép cho khoản gửi một lần:
$$\boxed{A = P \times (1 + r)^n}$$

\vspace{0.5cm}
\textbf{Bài toán 2:} Mỗi tháng gửi đều đặn 400.000 đồng vào ngân hàng với lãi suất 0,583\%/tháng. Tính số tiền nhận được sau 24 tháng.
\begin{itemize}[leftmargin=*,noitemsep]
\item Số tiền gửi tháng cuối cùng (tháng 24): 400.000
\item Số tiền gửi tháng 23: $400.000 \times (1 + 0,00583)$
\item Số tiền gửi tháng 22: $400.000 \times (1 + 0,00583)^2$
\item $\ldots$
\item Tổng số tiền là tổng của một Cấp số nhân $S_n = u_1 \times \frac{q^n - 1}{q-1}$
\end{itemize}
\textbf{Kết luận:} Công thức lãi kép cho khoản gửi định kỳ:
$$\boxed{A = P \times \frac{(1+r)^n-1}{r}}$$
Trong đó $P$ là số tiền gửi mỗi kỳ, $r$ là lãi suất theo kỳ, $n$ là số kỳ.
\end{phieuhoc_tap_box}

\newpage
% CẬP NHẬT: PHT 2 phải khớp với 3 tab của app
\subsection*{Phụ lục 2: Phiếu học tập số 2 - Hướng dẫn sử dụng ứng dụng web}
\begin{phieuhoc_tap_box}{Phụ lục 2: Phiếu học tập số 2 - Nhiệm vụ với Web App \texttt{CongCuTuVanTietKiem.html}}
\textbf{\color{accent}I. Giới thiệu ứng dụng}
\begin{itemize}[leftmargin=*,noitemsep]
\item \textbf{Tên ứng dụng:} Công cụ tư vấn tiết kiệm tương lai
\item \textbf{Địa chỉ truy cập:} File \texttt{CongCuTuVanTietKiem.html} (mở bằng trình duyệt)
\item \textbf{Cấu trúc 3 Tab:}
    \begin{itemize}
    \item \textbf{Tab 1 - TÍNH TOÁN:} Nhập tham số, tính kết quả, xem biểu đồ tròn, bật/tắt Lạm phát.
    \item \textbf{Tab 2 - PHÂN TÍCH KỊCH BẢN:} Thêm/xóa kịch bản, so sánh bằng biểu đồ cột, xem phân tích độ nhạy (biểu đồ đường).
    \item \textbf{Tab 3 - HỌC TẬP:} Đọc công thức Toán, tìm hiểu kênh đầu tư khác, xem quy trình STEM.
    \end{itemize}
\end{itemize}
\textbf{\color{accent}II. Hướng dẫn thực hiện nhiệm vụ}
\begin{itemize}[leftmargin=*,noitemsep]
\item \textbf{Bước 1: Mở Tab "TÍNH TOÁN"}
    \begin{itemize}
    \item Nhập các tham số của bài toán gốc:
    \item \textit{Số tiền hiện có:} 5.000.000
    \item \textit{Số tiền gửi thêm mỗi tháng:} 400.000
    \item \textit{Lãi suất năm (\%):} 7
    \item \textit{Thời gian (tháng):} 24
    \item \textit{Ghi lại kết quả (Tổng tiền):} \underline{\hspace{4cm}} VNĐ
    \item \textit{Ghi lại Trạng thái mục tiêu (20tr):} \underline{\hspace{4cm}}
    \item \textbf{Thử nghiệm:} Bật chức năng "Lạm phát" (3.5\%). Ghi lại "Giá trị thực": \underline{\hspace{3cm}} VNĐ
    \end{itemize}
\item \textbf{Bước 2: Mở Tab "PHÂN TÍCH KỊCH BẢN"}
    \begin{itemize}
    \item Ứng dụng đã có sẵn 3 kịch bản. Hãy nhấn "Thêm kịch bản" để tạo \textbf{Kịch bản 4} của nhóm em.
    \item \textit{Mục tiêu:} Tạo kịch bản để đạt được 20.000.000 VNĐ.
    \item \textit{Kịch bản 4 đề xuất:} (VD: Tăng tiền gửi tháng lên 600.000) \underline{\hspace{6cm}}
    \item Quan sát biểu đồ cột. Kịch bản nào tốt nhất? \underline{\hspace{4cm}}
    \end{itemize}
\item \textbf{Bước 3: Phân tích Độ nhạy (vẫn ở Tab 2)}
    \begin{itemize}
    \item Kéo xuống xem biểu đồ đường "Phân tích độ nhạy cảm".
    \item \textit{Nhận xét:} Yếu tố nào (Lãi suất hay Tiền gửi tháng) làm cho các đường đồ thị dốc hơn (ảnh hưởng mạnh hơn)? \underline{\hspace{10cm}}
    \end{itemize}
\item \textbf{Bước 4: Mở Tab "HỌC TẬP"}
    \begin{itemize}
    \item Đọc phần "Phân tích đầu tư". Kể tên 2 kênh đầu tư khác ngoài gửi tiết kiệm:
    \item 1: \underline{\hspace{4cm}} | 2: \underline{\hspace{4cm}}
    \end{itemize}
\end{itemize}
\end{phieuhoc_tap_box}

\newpage
\subsection*{Phụ lục 3: Rubric đánh giá sản phẩm}
% SỬA LỖI TRÀN LỀ
\begin{tabularx}{\linewidth}{@{} >{\raggedright\arraybackslash}p{3.5cm} >{\raggedright\arraybackslash}X >{\raggedright\arraybackslash}X >{\raggedright\arraybackslash}X @{}}
    \toprule
    \rowcolor{TableHeader}
    \sffamily\bfseries Tiêu chí & \sffamily\bfseries Xuất sắc (4 điểm) & \sffamily\bfseries Khá (3 điểm) & \sffamily\bfseries Cần cải thiện (1-2 điểm) \\
    \midrule
    
    \textbf{Tính chính xác của công thức toán học} 
    & 
    % SỬA LỖI: Dùng \begin{itemize}
    \begin{itemize}[leftmargin=*,noitemsep,topsep=1ex, partopsep=0pt]
        \item Công thức (PHT 1) chính xác
        \item Giải thích rõ ràng nguồn gốc
        \item Ứng dụng linh hoạt
    \end{itemize}
    & 
    \begin{itemize}[leftmargin=*,noitemsep,topsep=1ex, partopsep=0pt]
        \item Công thức chủ yếu chính xác
        \item Giải thích cơ bản
        \item Ứng dụng được
    \end{itemize}
    & 
    \begin{itemize}[leftmargin=*,noitemsep,topsep=1ex, partopsep=0pt]
        \item Công thức sai
        \item Không giải thích được
        \item Không ứng dụng được
    \end{itemize} \\
    \addlinespace
    % CẬP NHẬT: Tiêu chí khớp với app
    \textbf{Khả năng sử dụng công nghệ (Web App)} 
    & 
    \begin{itemize}[leftmargin=*,noitemsep,topsep=1ex, partopsep=0pt]
        \item Sử dụng thành thạo cả 3 tabs
        \item Phân tích đúng (Lạm phát, Độ nhạy)
    \end{itemize}
    & 
    \begin{itemize}[leftmargin=*,noitemsep,topsep=1ex, partopsep=0pt]
        \item Sử dụng tốt Tab 1, 2
        \item Còn lúng túng ở Tab 3, Độ nhạy
    \end{itemize}
    & 
    \begin{itemize}[leftmargin=*,noitemsep,topsep=1ex, partopsep=0pt]
        \item Chỉ dùng được Tab 1
        \item Không phân tích được
    \end{itemize} \\
    \addlinespace
    \textbf{Tính khả thi của phương án (PHT 2)} 
    & 
    \begin{itemize}[leftmargin=*,noitemsep,topsep=1ex, partopsep=0pt]
        \item Đề xuất 2+ phương án thực tế, khả thi
        \item Phân tích chi tiết ưu nhược điểm
    \end{itemize}
    & 
    \begin{itemize}[leftmargin=*,noitemsep,topsep=1ex, partopsep=0pt]
        \item Đề xuất 1-2 phương án
        \item Phân tích cơ bản
    \end{itemize}
    & 
    \begin{itemize}[leftmargin=*,noitemsep,topsep=1ex, partopsep=0pt]
        \item Phương án không khả thi
        \item Thiếu phân tích
    \end{itemize} \\
    \addlinespace
    \textbf{Khả năng trình bày và bảo vệ ý tưởng} 
    & 
    \begin{itemize}[leftmargin=*,noitemsep,topsep=1ex, partopsep=0pt]
        \item Trình bày mạch lạc, logic
        \item Hỗ trợ trực quan hiệu quả
        \item Trả lời trọn vẹn
    \end{itemize}
    & 
    \begin{itemize}[leftmargin=*,noitemsep,topsep=1ex, partopsep=0pt]
        \item Trình bày rõ ràng
        \item Có hỗ trợ trực quan
        \item Trả lời được
    \end{itemize}
    & 
    \begin{itemize}[leftmargin=*,noitemsep,topsep=1ex, partopsep=0pt]
        \item Trình bày rời rạc
        \item Thiếu trực quan
        \item Không trả lời được
    \end{itemize} \\ % SỬA LỖI: Xóa \end{tcolorbox}
    \bottomrule
\end{tabularx}
\vspace{1em}
\noindent\textbf{Tổng điểm:} \underline{\hspace{2cm}}/16 điểm\\
\textbf{Nhận xét của giáo viên:}\\
\vspace{3cm}

\subsection*{Phụ lục 4: Bảng tham khảo lãi suất tiết kiệm}
\begin{center}
% Nâng cấp: Dùng booktabs
\begin{tabularx}{0.9\linewidth}{@{} l C C C C @{}}
\toprule
\header{Ngân hàng} & \header{Không kỳ hạn} & \header{6 tháng} & \header{12 tháng} & \header{24 tháng} \\
\midrule
Vietcombank & 0,5\% & 5,3\% & 7,0\% & 7,5\% \\
Techcombank & 0,6\% & 5,5\% & 7,2\% & 7,8\% \\
VPBank & 0,4\% & 5,2\% & 6,8\% & 7,3\% \\
MB Bank & 0,5\% & 5,4\% & 7,1\% & 7,6\% \\
\bottomrule
\multicolumn{5}{@{}l}{\footnotesize * Số liệu cập nhật tháng 10/2025, mang tính minh họa.}
\end{tabularx}
\end{center}

\newpage
% CẬP NHẬT: Phụ lục 5 khớp với file HTML
\subsection*{Phụ lục 5: Mã nguồn ứng dụng web (tham khảo)}
\textbf{Giới thiệu:} Ứng dụng được phát triển bằng \textbf{HTML, CSS và JavaScript (vanilla JS)}, không sử dụng React.js.
% SỬA LỖI: Dùng \itemize
\begin{itemize}[leftmargin=*,noitemsep]
\item \textbf{Cấu trúc 3 Tabs:} (1) Tính toán, (2) Phân tích kịch bản, (3) Học tập.
\item \textbf{Thư viện sử dụng:} \texttt{Chart.js} (vẽ biểu đồ), \texttt{SweetAlert2} (hiển thị thông báo).
\item \textbf{Đặc điểm:} Chạy trực tiếp từ file \texttt{.html}, không cần server, tương thích Quyết định 791/QĐ-SGDĐT.
\end{itemize}

\textbf{Mã nguồn mẫu - Hàm tính toán chính (JavaScript - trích từ file HTML):}
% CẬP NHẬT: Mã nguồn khớp với file HTML
\begin{verbatim}
// Main calculation function
function calculate() {
    try {
        const initial = parseFloat(document
        .getElementById('initial').value) || 0;
        const monthly = parseFloat(document
        .getElementById('monthly').value) || 0;
        const rateAnnual = parseFloat(document
        .getElementById('rate').value) || 0;
        const months = parseInt(document
        .getElementById('monthsInput').value) || 1;
        const isUsingInflation = document
        .getElementById('inflationToggle').checked;
        const inflationRate = isUsingInflation ? 
            (parseFloat(document
            .getElementById('inflationRate')
            .value) || 3.5) / 100 : 0;
        
        // Convert annual rate to monthly rate
        const rateMonthly = rateAnnual / 100 / 12;
        
        // Calculate compound interest for initial deposit
        const initialFutureValue = initial * Math
        .pow(1 + rateMonthly, months);
        
        // Calculate compound interest for monthly deposits
        let monthlyFutureValue = 0;
        if (rateMonthly > 0) {
            monthlyFutureValue = monthly * ((Math
            .pow(1 + rateMonthly, months) - 1) 
                / rateMonthly);
        } else {
            monthlyFutureValue = monthly * months;
        }
        
        // Total amount before inflation
        const nominalTotal = initialFutureValue 
        + monthlyFutureValue;
        
        // Calculate real value after inflation
        const inflationFactor = isUsingInflation ? 
            Math.pow(1 + inflationRate/12, months) : 1;
        const realTotal = nominalTotal / inflationFactor;
        
        // ... (phần còn lại của hàm cập nhật giao diện) ...
        
        return {
            initial, monthly, rateAnnual, months, nominalTotal, 
            realTotal, /* ... */
        };
    } catch (error) {
        console.error('Lỗi tính toán:', error);
        // ... (xử lý lỗi) ...
    }
}
\end{verbatim}

\textbf{Tài nguyên học tập:}
\begin{itemize}[leftmargin=*,noitemsep]
\item \textbf{Mã nguồn đầy đủ:} Có thể xem trực tiếp bằng cách mở file \texttt{CongCuTuVanTietKiem.html} bằng trình soạn thảo văn bản (Notepad++, VS Code...) hoặc chọn "View Page Source" (Xem Nguồn Trang) trên trình duyệt.
\end{itemize} % SỬA LỖI
\textbf{Hướng dẫn cài đặt cho giáo viên:}
\begin{itemize}[leftmargin=*,noitemsep]
\item Không cần cài đặt.
\item Chỉ cần gửi file \texttt{CongCuTuVanTietKiem.html} cho học sinh.
\item Yêu cầu học sinh mở file bằng trình duyệt web (Chrome, Firefox, Edge...).
\end{itemize} % SỬA LỖI
\end{competitionsection}

\end{document}